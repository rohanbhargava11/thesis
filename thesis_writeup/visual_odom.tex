%% LyX 2.0.3 created this file.  For more info, see http://www.lyx.org/.
%% Do not edit unless you really know what you are doing.
\documentclass[english]{article}
\usepackage[T1]{fontenc}
\usepackage[latin9]{inputenc}
\usepackage{graphicx}

\makeatletter

%%%%%%%%%%%%%%%%%%%%%%%%%%%%%% LyX specific LaTeX commands.
%% A simple dot to overcome graphicx limitations
\newcommand{\lyxdot}{.}


\makeatother

\usepackage{babel}
\begin{document}
For our sensor model to work i.e. assign weights to our particles
we need some sort of reference. Some examples are Landmarks, Maps
etc. Austin and ELizar used static maps as a reference. We have a
side sonar on our AUVs and image generated can be used as a reference.
The side sonar outputs pings which can be used to generate images.
Andrew Vardy explains on how to generate images and in our algorithm
we use it as a black box. The final image that is generated is shown
below. 

\begin{figure}


\caption{Side Sonar Image}


\includegraphics{\lyxdot \lyxdot /visual_sonar/sonar_59/18}

\end{figure}


We run feature extraction such as SURF on the image to detect interest
points. The circle in the image shown below are the interest points. 

\begin{figure}


\caption{SURF Keypoints}


\includegraphics{\lyxdot \lyxdot /visual_sonar/surf_working}

\end{figure}


These interest points can be treated as landmarks. We use a high Hessian
threshold so that we have a maximum of 4 landmarks. We can measure
the distance of the robot and the particles from the landmark. Based
on the measurements we assign weights to the particle. This method
allows us independence from a static map as well as helps in learning
the motion model on the fly in a new environment. 
\end{document}
