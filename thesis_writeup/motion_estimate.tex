%% LyX 2.0.3 created this file.  For more info, see http://www.lyx.org/.
%% Do not edit unless you really know what you are doing.
\documentclass[english]{article}
\usepackage[T1]{fontenc}
\usepackage[latin9]{inputenc}
\usepackage{babel}
\begin{document}
In AUV the motion estimation is generally done through dead-reckoning.
It is a process of calculating the current position based upon the
previous position and the speed of vessel. The velocity of the AUV
can be estimated by the acceleration measurement supplied by an Inertial
Measurement Unit. Another way is to use a Doppler Velocity Log that
measures velocity in water by measuring the Doppler effect on scattered
sound waves. Dead reckoning may have significant errors as the velocity
and direction must be accurately known at every time step. As the
next estimate is based on the previous estimate the error accumulates. 

In this algorithm we propose to get motion estimates using side sonar
images. As stated before we run SURF on the images and generate some
key points. These key points are matched in the next image using a
knn based matcher. The matched key points gives us an estimate on
the movement of the vehicle. In the figure we show two consecutive
images and in the first image we have the key points marked in circle.
In the next image we have the matched key points marked in green circles. 

This estimate can be coupled with the previous estimate to calculate
the current position. The visual input to the dead reckoning algorithm
has its pros and cons.The main advantage of using a visual estimate
is that it doesn't suffer from drift which is concern for underwater
vehicles. The disadvantage lies in the fact that we don't have side
sonar images available every time. We can solve the problem by combining
the visual input with the velocity estimates. We can pass the visual
motion estimate and the DVL estimate to a Kalman Filter and use the
output as an input to our dead-reckoning estimate. The second disadvantage
is the computation power available on AUV. To specifically deal with
the problems we use a high Hessian threshold to extract maximum of
4 landmarks so that feature matching is not computationally expensive. 

We validate our algorithm on datasets consisting of side sonar images
and the total distance the AUV moves. 
\end{document}
