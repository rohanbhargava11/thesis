%% LyX 2.0.3 created this file.  For more info, see http://www.lyx.org/.
%% Do not edit unless you really know what you are doing.
\documentclass[english]{article}
\usepackage[T1]{fontenc}
\usepackage[latin9]{inputenc}
\usepackage{graphicx}

\makeatletter

%%%%%%%%%%%%%%%%%%%%%%%%%%%%%% LyX specific LaTeX commands.
%% Because html converters don't know tabularnewline
\providecommand{\tabularnewline}{\\}
%% A simple dot to overcome graphicx limitations
\newcommand{\lyxdot}{.}


\makeatother

\usepackage{babel}
\begin{document}
We use a simulation to demonstarte the effictivnes of learning the
motion model. The simulation mainly consists of particle filter SLAM.
We use a motion model described in the above chapters. The sensor
model basically measures the distance from the four static landmarks
defined at the start of the experiment. The experiment runs over 100
time steps and at every 5\textsuperscript{th} time step we change
the paramters of the motion model. As described in the above chapters
the parameters that we intendend to learn are $\sigma_{D_{d}}^{2}$,$\sigma_{T_{d}}^{2}$,$\sigma_{D_{1}}^{2},\sigma_{D_{r}}^{2}$,$\sigma_{T_{r}}^{2}$,$\sigma_{D_{1}}^{2}$,$\sigma_{T_{1}}^{2}$.
In the first stage of the experiment at every 5\textsuperscript{th }timestep
we change $\sigma_{D_{d}}^{2}$ or $\sigma_{T_{r}}^{2}$in our motion
model. The following table and plots describes the five experiments
that were conducted in the simulation.

\begin{tabular}{|c|c|c|c|c|c|}
\hline 
No. & $\sigma_{D_{d}}$ & $\sigma_{T_{r}}$ & $\sigma_{D_{d}}^{*}$ & $\sigma_{T_{r}}^{*}$ & Sensor Noise\tabularnewline
\hline 
\hline 
1 & 0.05 & 0.05 & 0.2 & 0.05 & 2.0\tabularnewline
\hline 
2 & 0.05 & 0.05 & 0.2 & 0.05 & 5.0\tabularnewline
\hline 
4 & 0.05 & 0.05 & 0.5 & 0.05 & 2.0\tabularnewline
\hline 
5 & 0.05 & 0.05 & 0.5 & 0.05 & 5.0\tabularnewline
\hline 
6 & 0.05 & 0.05 & 0.05 & 0.2 & 5.0\tabularnewline
\hline 
7 & 0.05 & 0.05 & 0.05 & 0.5 & 5.0\tabularnewline
\hline 
\end{tabular}

$\sigma_{D_{d}}$,$\sigma_{T_{r}}$ are the parameters values that
the motion model was initalized. These values are altered in order
to simulate a change in the motion model and they are descibed by
$\sigma_{D_{d}}^{*}$,$\sigma_{T_{r}}^{*}$. The sensor noise can
be descibed as the confidence the robot has in its sensor model .
The impact of the noise on the locilization error can be seen in the
following plots.

\begin{figure}
\caption{$\sigma_{D_{d}}$=0.05 $\sigma_{D_{d}}^{*}$ = 0.2 Sensor Noise= 2.0}


\includegraphics[scale=0.25]{\lyxdot \lyxdot /plot_motion_model/100_0\lyxdot 05_0\lyxdot 2_1_2\lyxdot 0_motion_model_1}
\end{figure}


\begin{figure}
\caption{$\sigma_{D_{d}}$=0.05 $\sigma_{D_{d}}^{*}$ = 0.2 Sensor Noise= 5.0}


\includegraphics[scale=0.25]{\lyxdot \lyxdot /plot_motion_model/100_0\lyxdot 05_0\lyxdot 2_1_5\lyxdot 0_motion_model_1}

\end{figure}


Figure 1 and Figure 2 are plots of the locilization error with different
sensor noises. We can see in both the cases the learned motion model
performed better than the static motion model. Another important point
is that the average error is less when the sensor noise is 2.0 as
compared to the second case. This can be accounted for the fact that
our locilization algorithm is more confident on the sensor model as
compared to the motion model. 

\begin{figure}
\caption{$\sigma_{D_{d}}$=0.05 $\sigma_{D_{d}}^{*}$ = 0.5 Sensor Noise= 2.0}


\includegraphics[scale=0.25]{\lyxdot \lyxdot /plot_motion_model/100_0\lyxdot 05_0\lyxdot 5_1_2\lyxdot 0_motion_model}

\end{figure}


\begin{figure}
\caption{$\sigma_{D_{d}}$=0.05 $\sigma_{D_{d}}^{*}$ = 0.5 Sensor Noise= 5.0}


\includegraphics[scale=0.25]{\lyxdot \lyxdot /plot_motion_model/100_0\lyxdot 05_0\lyxdot 5_1_5\lyxdot 0_motion_model_1}

\end{figure}


As we can see in Figure 3 and Figure 4 at 5\textsuperscript{th} time
step the error shooting up but the learned motion model brings back
the error whereas the static motion model takes time to recover back
depending upon the sensor noise. In both the figures we can see that
the error is pretty static in the learned motion model whereas in
the static motion model there is a lot of fluctulation. 

\begin{figure}
\caption{$\sigma_{T_{r}}$=0.05 $\sigma_{T_{r}}^{*}$ = 0.2 Sensor Noise= 5.0}


\includegraphics[scale=0.75]{\lyxdot \lyxdot /plot_motion_model/100_0\lyxdot 05_0\lyxdot 2_rotation_5\lyxdot 0_motion_model}

\end{figure}


\begin{figure}
\caption{$\sigma_{T_{r}}$=0.05 $\sigma_{T_{r}}^{*}$ = 0.5 Sensor Noise= 5.0}


\includegraphics[scale=0.25]{\lyxdot \lyxdot /plot_motion_model/100_0\lyxdot 05_0\lyxdot 5_rotation_5\lyxdot 0_motion_model}

\end{figure}


Figure 5 and Figure 6 describe the errors when the robot rotational
motion is much more than the translational motion. We can clearly
see that the learned motion model quickly adapts to the changes whereas
the static motion model struggles to get the error down. 

In all the cases it was very clear that we could see the adaptive
motion model performing better than the static. The sensor noise had
its imapct on the overall error. Robot calibration is important to
process in mobile robotics. The proposed algoirhtm is an automated
process which can help us in better navigation of the robots and can
be used for any motion model. 
\end{document}
