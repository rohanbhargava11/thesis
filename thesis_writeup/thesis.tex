% simple.tex - simple Master's thesis sample
% $Id: simple.tex 309 2011-01-28 14:46:48Z vlado $
\documentclass[12pt]{dalcsthesis}
% to prepare draft version use option draft:
%\documentclass[12pt,draft]{dalcsthesis}

\begin{document}
\mcs  % options are \mcs, \macs, \mec, \mhi, \phd, and \bcshon
\title{Feature Based adaptive motion model}
\author{Rohan Bhargava}
\defenceday{1}
\defencemonth{October}
\defenceyear{2013}
\convocation{January}{2014}

% Use multiple \supervisor commands for co-supervisors.
% Use one \reader command for each reader.

\supervisor{Dr. Thomas Trappenberg}
\reader{D. Odaprof}
\reader{A. External}

\nolistoftables
\nolistoffigures

\frontmatter

\begin{abstract}
This is a test document.
\end{abstract}

\begin{acknowledgements}
Thanks to all the little people who make me look tall.
\end{acknowledgements}

\mainmatter

\chapter{Introduction}
You need to read the papers on learning the motion model using EM. There you can get a hint of introduction for the paper
Get it done!  Use reference material by Lamport~\cite{latex-by-lamport} or
Gooses, Mittelback, and Samarin~\cite{latex-companion}.

\chapter{Doing It}

\section{Getting Ready}

Get all the parts that I need.  I can throw in a whole pile of terms like
preparation,
methodology,
forethought,
and
analysis
as examples for me to use in the future.

\section{Next Step}

Do it!

Of course, you have to have pictures to show how you did it to make people
understand things better.

\chapter{Conclusion}

Did it!

\bibliographystyle{plain}
\bibliography{simple}

\end{document}
